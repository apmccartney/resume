\documentclass[legalpaper]{article}
\usepackage{standalone}
\usepackage[utf8]{inputenc}            % A package that expands the character set to UTF-8
\usepackage[english]{babel}            % A package that describes the language use for latex commands
\usepackage[margin=1.0cm]{geometry}    % A package which adjusts the page margins to 1.5 cm
\usepackage{sectsty}                   % A package which allows a user to edit the section heading properties
\usepackage{hyperref}                  % A package for embedding hypertext links
\usepackage{enumitem}
\usepackage{framed, color}
\usepackage{multicol}
\usepackage{mdwlist}
\usepackage{amssymb}

\newlength\tindent
\setlength{\tindent}{\parindent}
\setlength{\parindent}{0pt}
\renewcommand{\indent}{\hspace*{\tindent}}

%\setlist[itemize]{topsep=0pt}
\renewcommand{\labelitemi}{\tiny$\blacksquare$}
\definecolor{shadecolor}{rgb}{0.035,0.361, 0.453}

\begin{document}
\setlength\multicolsep{0pt}
\pagestyle{empty}

\begin{center}
  {\Large \textbf{Austin McCartney}\\}
  3200 Canyon RD Apartment 1104 $\vert$ Los Alamos, NM $\vert$ +1-352-354-2756 $\vert$ austin.p.mccartney@gmail.com
\end{center}

\begin{center}
  \begin{shaded}
    { \color{white} \textbf{EXPERIENCE}}
  \end{shaded}
\end{center}

\textbf{Post-Master's Appointment}\\
Los Alamos National Laboratory \hfill May 2015 - Present\\
~\\
During my appointment, I contributed to and developed several software products related to nuclear science, including 
\medskip
\begin{itemize}[noitemsep, before={\vspace*{-0.5\baselineskip}}]
\item the NJOY21 driver and command line interface.
\item C++ hooks to NJOY2012 Fortran routines.
\item the ACEtk C++ library.
\item functionality to collapse multigroup neutron and photon data using user-specified group weights in the NDI C99 code
\item a collection of reusable CMake scripts supporting arbitrary composition and configuration of component-based libraries.
\item a python script providing dependency management for NJOY21 component libraries.
\end{itemize}

As part of the migration of NJOY21 component libraries to GitHub, I have implemented
\medskip
\begin{itemize}[noitemsep, before={\vspace*{-0.5\baselineskip}}]
\item automated continuous integration on Linux and OSX platforms using the TravisCI service.
\item automated test coverage evaluation and monitering using the Gcov tool and the Coveralls Service.
\item integration to the Coverity static analysis suite.
\end{itemize}

My collaborations during this time include
\medskip
\begin{itemize}[noitemsep, before={\vspace*{-0.5\baselineskip}}]
\item a refactor of the RTran discrete ordinates neutron transport program, achieving a 300x speed up on problems of interest
\item several components to the Charged Particle Tools (CPT) library.
\end{itemize}

Recently, I was one of six named as a LANL 2016 Distinguished Student.

\begin{center}
  \begin{shaded}
    { \color{white} \textbf{EDUCATION}}
  \end{shaded}
\end{center}
\textbf{Georgia Institute of Technology} (Atlanta, GA)\hrule \vspace{1mm}

Master of Science in Nuclear Engineering \hfill August 2013 - May 2015\\
\vspace{\baselineskip} %~\\
~~~Program Highlights
\begin{itemize}[topsep=0pt, before={\vspace*{-0.5\baselineskip}}]%[topsep=-20pt]%[nolistsep]
\item High-Performance Computing: Tools and Applications\\
  Course projects revolved around parallel programming models (Cilk Plus, OpenMP, MPI, Nvidia CUDA, and SIMD Compiler Intrinsics) and tools for analyzing and tuning performance, parallelism, and memory access.

  As a final project, developed a sparse matrix storage scheme for matrices resulting from finite difference discretization of PDEs to achieve high throughput memory access during matrix-vector multiplication operations on Nvidia GPGPUs.

\item Reactor Physics\\
  This course emphasized the implementation of solution methods for systems of linear differential equations in C and Fortran.
  As a course project, developed a finite difference 3-dimensional, multigroup neutron diffusion code in Fortran 95.

\item Finite Element Method: Theory and Practice\\
  Provided an in-depth understanding of the theory and formulation of various finite elements with an emphasis on application to mechanical engineering.
  ANSYS finite element software was taught as part of the lab practicum.

\end{itemize}
~\\
\textbf{University of Florida} (Gainesville, FL) \hrule \vspace{1mm}
Bachelor of Science in Nuclear Engineering \hfill January 2009 - December 2011\\
\vspace{\baselineskip} %~\\
~~~Program Highlights
\begin{itemize}[topsep=0pt, before={\vspace*{-0.5\baselineskip}}]%[noitemsep,nolistsep]
\item Special Problems\\
  Conducted fluid mechanics research under the supervision of Dr.\@ DuWayne Schubring.
  Developed a statistical model in MATLAB describing the relationships between quantities of interest in disturbance waves in vertical annular two phase flow.
\end{itemize}
~\\
%\textbf{College of Central Florida} (Ocala, FL) \hrule
%Associate of the Arts in Mathematics \hfill August 2007 - December 2008\\
%~\\
\textbf{Independent Coursework}\hrule \vspace{1mm}
% \textbf{Udacity}
% \begin{multicols}{2}
%   \begin{itemize}[noitemsep,nolistsep]
%     \renewcommand{\labelitemi}{\tiny$\blacksquare$}
%   \item Data Analysis with R
%   \item Differential Equations in Action
%   \item Introduction to Computer Science
%   \item Introduction to Data Science
%   \item Introduction to Parallel Programming
%   \item Version Control for Code
%   \end{itemize}
% \end{multicols}
\vspace{\baselineskip} %~\\
\textbf{Coursera}

\begin{itemize}[noitemsep,nolistsep]

\item Johns Hopkins University
  \begin{multicols}{2}
    \begin{description*}
      \renewcommand{\labelitemi}{\tiny$\blacksquare$}
    \item[] Computing for Data Analysis
    \item[] Getting and Cleaning Data
    \item[] R Programming
    \item[] The Data Scientist's Toolbox
    \end{description*}
  \end{multicols}
  
  \vspace{2mm}
  
  \begin{multicols}{2}
  \item Rice University\\
    Introduction to Interactive Programming in Python
  \item University of Illinois at Urbana–Champaign\\
    Heterogeneous Parallel Programming
  \end{multicols}

\end{itemize}
\vspace{\baselineskip} %~\\
\textbf{National Energy Research Scientific Computing Center}
\begin{itemize}[noitemsep,nolistsep]
\item Object-Oriented Programming in Fortran 2003
\end{itemize}

\begin{center}
  \begin{shaded}
    { \color{white} \textbf{ TECHNICAL SKILLS}}
  \end{shaded}
\end{center}
\begin{tabular}{l p{5in}}
  ~\textbf{Computer Programming}: &  C (99), C++ (03, 11, 14), Fortran (90/95, 2003/08), Python (2.7, 3.4),\\
  & Java, Rust, R, MATLAB/Octave, Bash\\
  \rule{0pt}{4ex} \textbf{Parallel Platforms}:   &  Nvidia CUDA, MPI, OpenMP, Cilk Plus\\
  \rule{0pt}{4ex} \textbf{Software Development Tools}:    &  Git, CMake, Catch Unit Testing Framework, Gcov Test Coverage Tool,\\
  & Valgrind Dynamic Analysis Framework, Doxygen Documentation Generator\\
  \rule{0pt}{4ex} \textbf{Operating Systems}:    &  Linux, OSX, Windows\\
  \rule{0pt}{4ex} \textbf{Miscellaneous}:        &  UML, Markdown, \LaTeX, Microsoft Office Suite\\
\end{tabular}



%\end{framed}
\end{document}

%sagemathcloud={``zoom_width'':100}

%%% Local Variables:
%%% mode: latex
%%% TeX-master: t
%%% End:
